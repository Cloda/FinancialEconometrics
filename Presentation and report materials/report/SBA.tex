\section{Structural Breaks Analysis}\label{sec:SBA}
    The analysis of the structural breaks in the Nelson-Siegel factor dynamics allows for the identification of periods of volatility or instability 
    in the bond market, which is crucial for investors and policymakers to make informed decisions. This section aims to detect structural breaks in 
    the Nelson-Siegel factor dynamics and examine the timing, magnitude, and nature of these breaks.

    
    Now let us examine the events that happened around the detected structural breaks. We will fix $m=4$ (corresponds to top-4 shocks for 20 years). 
    The detected structural breaks can be found in \cref{tab:detected_breakpoints_obs,tab:detected_breakpoints_dates}. In \cref{tab:optimal_partition_beta0,tab:optimal_partition_beta1,tab:optimal_partition_beta2}, you can find the full list of optimal $(m+1)$-segment partition for $\beta_0$, $\beta_1$, and $\beta_2$ dynamics, respectively.

    \begin{table}[!h]
        \centering
        \begin{tabular}{|c|c|c|c|c|}\hline
            TTM Scale               & First shock & Second shock & Third shock & Fourth shock \\\hline
            Long-run ($\beta_0$)    & 657         & 1384         & 3943        & 4502 \\
            Medium-run ($\beta_1$)  & 938         & 1495         & 2832        & 3524 \\
            Short-run ($\beta_2$)   & 1589        & 2252         & 2869        & 3115 \\\hline
        \end{tabular}
        \caption{Detected structural breaks, numbers of observation detected as breakpoints.}
        \label{tab:detected_breakpoints_obs}
    \end{table}

    \begin{table}[!h]
        \centering
        \begin{tabular}{|c|c|c|c|c|}\hline
            TTM Scale               & First shock & Second shock & Third shock & Fourth shock \\\hline
            Long-run ($\beta_0$)    & 2005-08-30  & 2008-08-08   & 2018-06-21  & 2020-09-09 \\
            Medium-run ($\beta_1$)  & 2006-10-17  & 2009-01-22   & 2014-01-21  & 2016-10-21 \\
            Short-run ($\beta_2$)   & 2009-06-09  & 2012-02-02   & 2014-03-14  & 2015-03-10 \\\hline
        \end{tabular}
        \caption{Detected structural breaks, dates.}
        \label{tab:detected_breakpoints_dates}
    \end{table}


    \begin{landscape}
        \pagestyle{empty}
        \begin{table}
            \centering
            \begin{tabular}{|c|ccccccccccccccccccc|}\hline
                $m$  &      &     &     &     &      &      &      &      &      &      &      &      &      &      &      &      &      &      &      \\\hline
                1  &      &     &     &     &      &      &      &      &      &      &      &      &      &      &      &      &      &      & 4619 \\
                2  &      &     &     &     &      & 1390 &      &      &      &      &      &      &      &      &      &      &      &      & 4619 \\
                3  &      &     & 657 &     &      & 1380 &      &      &      &      &      &      &      &      &      &      &      &      & 4619 \\
                4  &      &     & 657 &     &      & 1384 &      &      &      &      &      &      &      &      &      & 3943 &      & 4502 &      \\
                5  &      &     & 657 &     &      & 1384 &      &      &      &      &      &      &      &      &      & 3943 &      & 4376 & 4622 \\
                6  &      &     & 657 &     &      & 1391 & 1654 &      & 2277 &      &      &      &      &      &      & 3868 &      & 4502 &      \\
                7  &      &     & 657 &     &      & 1391 & 1654 &      & 2277 &      &      &      &      &      &      & 3943 &      & 4376 & 4622 \\
                8  &      &     & 657 &     &      & 1391 & 1654 &      & 2279 &      &      & 2966 &      &      &      & 3945 &      & 4376 & 4622 \\
                9  &      &     & 657 &     &      & 1391 & 1654 &      & 2279 &      &      & 2966 &      & 3525 &      & 3867 &      & 4376 & 4622 \\
                10 &      &     & 657 &     &      & 1391 & 1654 &      & 2279 &      &      & 2966 &      & 3370 & 3616 & 3945 &      & 4376 & 4622 \\
                11 &      &     & 659 &     & 1146 & 1392 & 1654 &      & 2279 &      &      & 2966 &      & 3370 & 3616 & 3945 &      & 4376 & 4622 \\
                12 &      &     & 657 &     &      & 1391 & 1654 &      & 2210 & 2465 & 2714 & 2966 &      & 3370 & 3616 & 3945 &      & 4376 & 4622 \\
                13 &      &     & 659 &     & 1146 & 1392 & 1654 &      & 2210 & 2465 & 2714 & 2966 &      & 3370 & 3616 & 3945 &      & 4376 & 4622 \\
                14 &  322 &     & 657 &     & 1146 & 1392 & 1654 &      & 2210 & 2465 & 2714 & 2966 &      & 3370 & 3616 & 3945 &      & 4376 & 4622 \\
                15 &  322 &     & 654 & 900 & 1146 & 1392 & 1654 &      & 2210 & 2465 & 2714 & 2966 &      & 3370 & 3616 & 3945 &      & 4376 & 4622 \\
                16 &  322 &     & 654 & 900 & 1146 & 1392 & 1654 & 1936 & 2210 & 2465 & 2714 & 2966 &      & 3370 & 3616 & 3945 &      & 4376 & 4622 \\
                17 &  322 &     & 654 & 900 & 1146 & 1392 & 1654 & 1936 & 2210 & 2465 & 2714 & 2966 &      & 3370 & 3616 & 3870 & 4116 & 4376 & 4622 \\
                18 &  322 &     & 654 & 900 & 1146 & 1392 & 1654 & 1936 & 2210 & 2465 & 2711 & 2958 & 3204 & 3450 & 3696 & 3942 & 4188 & 4434 & 4680 \\
                19 &  246 & 492 & 738 & 984 & 1230 & 1476 & 1722 & 1968 & 2214 & 2465 & 2711 & 2958 & 3204 & 3450 & 3696 & 3942 & 4188 & 4434 & 4680 \\
                \hline
            \end{tabular}
            \caption{Optimal $(m+1)$-segment partition for $\beta_0$ dynamics.}
            \label{tab:optimal_partition_beta0}
        \end{table}

        \begin{table}
            \centering
            \begin{tabular}{|c|ccccccccccccccccccc|}\hline
                $m$  &      &     &     &     &      &      &      &      &      &      &      &      &      &      &      &      &      &      &      \\\hline
                1  &      &     &     &     &      &      &      &      &      &      & 2814 &      &      &      &      &      &      &      &     \\
                2  &      &     & 672 &     &      &      &      &      &      &      & 2829 &      &      &      &      &      &      &      &     \\
                3  &      &     & 672 &     &      &      &      &      &      &      &      & 2832 &      & 3524 &      &      &      &      &     \\
                4  &      &     &     & 938 &      & 1495 &      &      &      &      &      & 2832 &      & 3524 &      &      &      &      &     \\
                5  &      &     &     & 938 &      & 1495 &      &      &      &      &      & 2832 &      & 3521 &      &      &      &      & 4619\\
                6  &      &     &     & 938 &      & 1495 &      &      &      &      &      & 2832 &      & 3524 &      &      &      & 4361 & 4607\\
                7  &      &     &     & 917 &      &      & 1772 &      & 2115 &      &      & 2832 &      & 3524 &      &      &      & 4361 & 4607\\
                8  &  246 &     & 672 &     &      &      & 1773 &      & 2115 &      &      & 2832 &      & 3524 &      &      &      & 4361 & 4607\\
                9  &  246 &     & 660 & 959 &      &      & 1772 &      & 2115 &      &      & 2832 &      & 3524 &      &      &      & 4361 & 4607\\
                10 &  246 &     & 660 & 992 &      & 1489 & 1775 &      & 2115 &      &      & 2832 &      & 3524 &      &      &      & 4361 & 4607\\
                11 &  246 &     & 660 & 992 &      & 1489 & 1775 &      & 2115 &      & 2751 & 2997 &      & 3521 &      &      &      & 4361 & 4607\\
                12 &  246 &     & 660 & 992 &      & 1489 & 1775 &      & 2115 &      & 2751 & 2997 &      & 3521 &      &      & 4080 & 4359 & 4605\\
                13 &  246 &     & 660 & 992 &      & 1489 & 1775 &      & 2115 &      & 2751 & 2997 &      & 3521 & 3767 &      & 4080 & 4359 & 4605\\
                14 &  246 &     & 660 & 992 &      & 1489 & 1775 &      & 2115 & 2503 & 2751 & 2997 &      & 3521 & 3767 &      & 4080 & 4359 & 4605\\
                15 &  246 &     & 660 & 992 &      & 1489 & 1775 &      & 2115 & 2503 & 2751 & 2997 & 3256 & 3521 & 3767 &      & 4080 & 4359 & 4605\\
                16 &  246 &     & 660 & 995 & 1243 & 1489 & 1775 &      & 2115 & 2503 & 2751 & 2997 & 3256 & 3521 & 3767 &      & 4080 & 4359 & 4605\\
                17 &  246 & 492 & 738 & 995 & 1243 & 1489 & 1775 &      & 2115 & 2503 & 2751 & 2997 & 3256 & 3521 & 3767 &      & 4080 & 4359 & 4605\\
                18 &  246 & 492 & 738 & 995 & 1243 & 1489 & 1765 & 2011 & 2257 & 2503 & 2751 & 2997 & 3256 & 3521 & 3767 &      & 4080 & 4359 & 4605\\
                19 &  246 & 492 & 738 & 984 & 1230 & 1476 & 1722 & 1968 & 2214 & 2460 & 2706 & 2952 & 3198 & 3444 & 3690 & 3936 & 4182 & 4428 & 4674\\
                \hline
            \end{tabular}
            \caption{Optimal $(m+1)$-segment partition for $\beta_1$ dynamics.}
            \label{tab:optimal_partition_beta1}
        \end{table}

        \begin{table}
            \centering
            \begin{tabular}{|c|ccccccccccccccccccc|}\hline
                $m$  &      &     &     &     &      &      &      &      &      &      &      &      &      &      &      &      &      &      &      \\\hline
                1  &       &     &     &      &      &      &      &      & 2252 &      &      &      &      &      &      &      &      &      &     \\
                2  &       &     &     &      &      & 1589 &      &      & 2223 &      &      &      &      &      &      &      &      &      &     \\
                3  &       &     &     &      &      &      &      &      & 2252 &      &      & 2869 & 3115 &      &      &      &      &      &     \\
                4  &       &     &     &      &      & 1589 &      &      & 2252 &      &      & 2869 & 3115 &      &      &      &      &      &     \\
                5  &   326 &     &     &      &      & 1589 &      &      & 2252 &      &      & 2869 & 3115 &      &      &      &      &      &     \\
                6  &   326 &     &     &      &      & 1589 &      &      & 2252 &      &      & 2869 & 3115 &      &      &      &      &      & 4605\\
                7  &   326 &     &     & 907  &      & 1451 &      &      & 2252 &      &      & 2869 & 3115 &      &      &      &      &      & 4605\\
                8  &   326 &     &     &      &      & 1589 &      &      & 2252 &      &      & 2868 & 3114 & 3374 & 3620 & 3866 &      &      &     \\
                9  &   326 &     &     & 907  &      & 1451 &      &      & 2252 &      &      & 2868 & 3114 & 3374 & 3620 & 3866 &      &      &     \\
                10 &   326 &     &     &      &      & 1589 &      &      & 2252 &      &      & 2868 & 3114 & 3374 & 3620 & 3911 & 4266 &      & 4605\\
                11 &   326 &     &     & 907  &      & 1451 &      &      & 2252 &      &      & 2868 & 3114 & 3374 & 3620 & 3911 & 4266 &      & 4605\\
                12 &   326 &     &     & 907  &      & 1451 &      & 2005 & 2252 &      &      & 2868 & 3114 & 3374 & 3620 & 3911 & 4266 &      & 4605\\
                13 &   326 &     &     & 907  &      & 1451 &      & 1977 & 2223 & 2503 &      & 2868 & 3114 & 3374 & 3620 & 3911 & 4266 &      & 4605\\
                14 &   326 &     &     & 907  &      & 1451 & 1697 & 1977 & 2223 & 2503 &      & 2868 & 3114 & 3374 & 3620 & 3911 & 4266 &      & 4605\\
                15 &   326 &     & 725 & 971  &      & 1451 & 1697 & 1977 & 2223 & 2503 &      & 2868 & 3114 & 3374 & 3620 & 3911 & 4266 &      & 4605\\
                16 &   326 & 572 &     & 908  & 1200 & 1451 & 1697 & 1977 & 2223 & 2503 &      & 2868 & 3114 & 3374 & 3620 & 3911 & 4266 &      & 4605\\
                17 &   326 & 572 &     & 908  & 1200 & 1451 & 1697 & 1977 & 2223 & 2503 &      & 2868 & 3114 & 3374 & 3620 & 3866 & 4112 & 4359 & 4605\\
                18 &   284 & 530 & 776 & 1022 & 1268 & 1514 & 1760 & 2006 & 2252 & 2503 &      & 2868 & 3114 & 3374 & 3620 & 3866 & 4112 & 4359 & 4605\\
                19 &   246 & 492 & 738 & 984  & 1230 & 1476 & 1722 & 1968 & 2214 & 2460 & 2706 & 2952 & 3198 & 3444 & 3690 & 3936 & 4182 & 4428 & 4675\\
                \hline
            \end{tabular}
            \caption{Optimal $(m+1)$-segment partition for $\beta_2$ dynamics.}
            \label{tab:optimal_partition_beta2}
        \end{table}
    \end{landscape}

    Now we shall begin the research of the historical events that happened around the detected structural breaks. 
    Note that the 'date' of the break is estimated approximately, therefore, there could be some discrepancies between 
    the actual date of the suggested event and the estimated date of the detected break.

    \subsection{Long-run yields}
            \paragraph{August 30, 2005} The detected structural break could be associated with three events:
            \begin{enumerate}
                \item The complete stabilization of the Russian economy after the 1998 crisis. In January 2005, free budget 
                balances in the amount of 218.4 billion rubles were transferred to the Stabfond. For the period January-November 
                of this year, the fund was replenished with revenues of the first quarter - in the amount of 210.8 billion rubles, 
                the second quarter - in the amount of 294.4 billion rubles, the third quarter - in the amount of 373.5 billion rubles, 
                October - by 137.7 billion rubles, November - 154.4 billion rubles. The Stabilization Fund began to be formed in Russia 
                on January 1, 2004 in order to reduce the risks associated with unfavorable foreign economic conditions, as well as a 
                tool for sterilizing excess money supply in circulation. It receives huge income of the budget from high oil prices. 
                See \cite{RBK2006}. This event could have led to the decrease in the future supply of the government bonds due to the possible 
                government budget proficit.
                \item The war in Iraq as an indirect cause due to the drastic increase in both spot and futures prices of oil, see 
                \cite{IMF2005}. At the time, the government budget in Russia was calculated using the international oil prices. The 
                increase in oil prices led to the proficit of the government budget, which, in fact, means that the government had more 
                money to spend on the economy (i.e. increase the foreign exchange reserves, reduce government debt and spend more for 
                unforeseen circumstances). Consequently, the emission of the government bonds was reduced, which led to the change in 
                the future supply of the bonds.
                \item United Nations Security Council resolution 1615, adopted unanimously on 29 July 2005, after reaffirming all 
                resolutions on Abkhazia and Georgia, particularly Resolution 1582 (2005), the council extended the mandate of the 
                United Nations Observer Mission in Georgia (UNOMIG) until 31 January 2006.
            \end{enumerate}
            \paragraph{August 08, 2008} The detected structural break could be (and probably is) associated with the start of the 
            Russo-Georgian Conflict and the beginning of the world finanical crisis. Both of these events had 
            a negative impact on the Russian economy. The conflict could have led to the increase in the government spending, which, 
            in turn, led to the increase in the government debt. The world financial crisis led to the decrease in the oil prices, 
            which, in turn, led to the decrease in the government budget proficit. The decrease in the proficit led to the increase 
            in the government debt. Both of these events led to the increase in the future supply of the government bonds. 
            \paragraph{June 21, 2018} The detected structural break could be associated with two events:
            \begin{enumerate}
                \item The Russian Federation experienced a prolonged period of widespread protests from March 2017 to the end of 
                2018, with a focus being on fighting corruption within the government and opposing the increase in the retirement age. 
                The aftermath of these protests could potentially include economic instability due to the disruption of normal business
                operations, increased government spending to address the demands of the protesters, and a loss of investor confidence in 
                the Russian economy. 
                \item 2018 FIFA World Cup. The economic aftermaths of the 2018 FIFA World Cup in Russia included a boost in tourism and 
                hospitality sectors, leading to increased consumer spending and infrastructure development. Additionally, the tournament 
                provided opportunities for small businesses and entrepreneurs, creating a positive impact on the local economy. However, 
                there were also concerns about the long-term utilization of the newly built infrastructure and the potential impact on 
                public finances due to the high costs of hosting the event. 
            \end{enumerate} 
            \paragraph{September 09, 2020} The detected structural break could be associated with first and second waves of COVID-19 
            pandemic. There was a sharp decline in global oil prices, a key source of revenue for Russia, leading to budgetary challenges. 
            Additionally, the pandemic-induced lockdowns and travel restrictions led to a contraction in economic activity, particularly 
            in sectors such as tourism and hospitality. The Russian government implemented several measures, including financial support 
            to businesses and individuals, to reduce the economic impact of the pandemic. The pandemic caused a recession and a range of 
            challenges for businesses and the workforce.

    \subsection{Medium-run yields}
            \paragraph{October 17, 2006} The detected structural break could be associated with two events:
            \begin{enumerate}
                \item The end of the Chechen War and the elimination of a number of militants in Chechnya: Basayev, the main representative 
                of Al-Qaeda in the North Caucasus, the Jordanian Sheikh Abu Omar Al-Seif, Maskhadov's successor, the so-called president 
                of Ichkeria Abdul Halim Saidulayev. The announcement of an amnesty, which, according to the latest data, was used by about 
                five hundred members of illegal armed groups. This could lead to the increase of the nationalistic sentiment, investor confidence, 
                international authority and stability of the economy.
                \item A series of high-profile murders - the first deputy chairman of the Central Bank of Russia, Andrei Kozlov (September 13), 
                the columnist of Novaya Gazeta, Anna Politkovskaya (October 7). This possibly led to the negative aftermaths.
            \end{enumerate}
            \paragraph{January 22, 2009} The detected structural break could be associated with the 2009 Russia-Ukraine gas dispute. In 
            2009, a dispute arose between Russian gas company Gazprom and Ukrainian gas company Naftogaz over accumulating debts for previous 
            gas supplies. The conflict led to a cutoff of Russian gas supplies to Ukraine, which in turn disrupted gas flows to Southeastern 
            Europe and parts of other European countries for 13 days. Despite attempts by the European Union to intervene, the crisis was not 
            resolved until January 18, when Russian Prime Minister Vladimir Putin and Ukrainian Prime Minister Yulia Tymoshenko negotiated a 
            new contract. Following the resolution, gas flows to Europe resumed, but both Russia and Ukraine suffered economic losses, and 
            their reputations as energy supplier and transit country were negatively impacted.
            \paragraph{January 21, 2014} The detected structural break could be associated with two major events: 
            \begin{enumerate}
                \item 2014 Winter Olympics. The economic aftermaths of the 2014 Winter Olympics in Russia included a boost in tourism and 
                hospitality sectors, leading to increased consumer spending and infrastructure development.
                \item The joining of Crimea into the Russian Federation. The event was followed by several economic sanctions imposed by 
                the United States and the European Union, which led to a decline in investor confidence and a reduction in foreign direct 
                investment. Additionally, the Russian economy was negatively impacted by the decline in oil prices and the depreciation 
                of the ruble. Furthermore, the 2014-2016 Russian economic crisis started.
            \end{enumerate}
            \paragraph{October 21, 2016} The detected structural break could be associated with two events:
            \begin{enumerate}
                \item Russian military operation in Syria. It had several impacts on the Russian economy. The defense budget increased significantly 
                to support military operations, diverting funds away from other sectors. However, the operation also led to an increase 
                in nationalistic sentiment, which potentially bolstered the mood of the investors. Overall, the operation 
                had mixed economic effects on the Russian economy.
                \item Allegations of institutionalized doping use by Russian athletes. The Russian doping scandal refers to a series of 
                allegations made against Russia by the World Anti-Doping Agency (WADA) concerning doping violations in multiple sports 
                in the country. The scandal had a negative impact on the reputation of Russian athletes and the country's sports industry, 
                leading to a decline in sponsorship deals and investments.
            \end{enumerate}

    \subsection{Short-run yields}
        We shall not describe the structural breaks for the short-run yields due to the volatile nature of the subject. However, we shall 
        note that the structural breaks for the short-run yields are mostly associated with the events that caused the structural breaks 
        for the medium-run yields.

    \subsection{Forecasting results}
        We decided to divide the dataset into 4 segments which we took from structural analysis of the long-run yields. The results of ARIMA and VAR models restricted to each sector could be found in \cref{tab:structuralARIMA,tab:structuralVAR}, respectively.
        \begin{table}[htbp]
            \centering
            \begin{tabular}{|l|l|l|l|l|l|l|}
            \hline
            Segment            & Factor    & MAPE       & ME       & MAE     & MPE         & RMSE    \\ \hline
            \multirow{4}{*}{0} & $\beta_0$ & 0.0083     & 6.6921   & 6.6921  & 0.0083      & 7.7627  \\ \cline{2-7} 
                               & $\beta_1$ & 0.0155     & 5.8167   & 7.4981  & -0.012      & 7.9825  \\ \cline{2-7} 
                               & $\beta_2$ & 0.1538     & 27.8315  & 30.2092 & 0.1455      & 47.6721 \\ \cline{2-7} 
                               & $\tau$    & 0.1499     & 0.2511   & 0.2511  & 0.1499      & 0.2786  \\ \hline
            \multirow{4}{*}{1} & $\beta_0$ & 0.0076     & -5.9872  & 5.9872  & 0.0076      & 6.2486  \\ \cline{2-7} 
                               & $\beta_1$ & 0.0275     & -5.15    & 6.8624  & 0.0209      & 7.2744  \\ \cline{2-7} 
                               & $\beta_2$ & 21312.0946 & -5.3942  & 55.4278 & -21312.0946 & 55.9851 \\ \cline{2-7} 
                               & $\tau$    & 0.7482     & -1.5317  & 1.5317  & -0.7482     & 1.7142  \\ \hline
            \multirow{4}{*}{2} & $\beta_0$ & 0.0578     & 48.5869  & 48.5869 & 0.0578      & 57.57   \\ \cline{2-7} 
                               & $\beta_1$ & 0.0972     & -15.7947 & 19.6762 & 0.0816      & 27.1274 \\ \cline{2-7} 
                               & $\beta_2$ & 0.1214     & 7.6326   & 27.0065 & -0.0172     & 29.0144 \\ \cline{2-7} 
                               & $\tau$    & 0.0916     & 0.3636   & 0.5129  & 0.069       & 0.5959  \\ \hline
            \multirow{4}{*}{3} & $\beta_0$ & 0.006      & -2.0011  & 5.5102  & -0.0021     & 6.0729  \\ \cline{2-7} 
                               & $\beta_1$ & 0.3471     & 23.9805  & 29.3697 & 0.2982      & 32.2907 \\ \cline{2-7} 
                               & $\beta_2$ & 0.5223     & -93.8611 & 93.8611 & -0.5223     & 95.6724 \\ \cline{2-7} 
                               & $\tau$    & 0.9588     & 1.8988   & 1.8988  & 0.9588      & 1.987   \\ \hline
            \end{tabular}
            \caption{ARIMA forecasting results for the structural NS factors.}
            \label{tab:structuralARIMA}
        \end{table}

        \begin{table}[htbp]
            \centering
            \begin{tabular}{|l|l|l|l|l|l|l|}
            \hline
            Segment            & Factor    & MAPE       & ME       & MAE     & MPE         & RMSE    \\ \hline
            \multirow{4}{*}{0} & $\beta_0$ & 0.0038     & 0.6743   & 3.053   & 0.0009      & 4.3991  \\ \cline{2-7} 
                               & $\beta_1$ &  0.017    &  5.8745   &   8.1959   &  -0.0121    &   8.7872   \\ \cline{2-7} 
                               & $\beta_2$ &   0.1034   &   15.4491  &   20.2472   &   0.0863   &   32.5685   \\ \cline{2-7} 
                               & $\tau$    &  0.1096    &   0.1865  & 0.1865     &  0.1096    &   0.196   \\ \hline

            \multirow{4}{*}{1} & $\beta_0$ &  0.0058    &  -0.3276   &   4.6186   &  -0.0004    &  5.1622    \\ \cline{2-7} 
                               & $\beta_1$ &  0.0355    & -8.4288    &  8.8186    &  0.0339    &  10.2788    \\ \cline{2-7} 
                               & $\beta_2$ &  26187.8876    &  -40.8754   &  47.3524    &  -26187.8563    & 57.5888     \\ \cline{2-7} 
                               & $\tau$    &  0.1294    &  0.0056   &  0.2309    &  0.0279    &  0.2566    \\ \hline

            \multirow{4}{*}{2} & $\beta_0$ &  2.4086    & 2062.9983    & 2062.9983     & 2.4086     &  2109.0292    \\ \cline{2-7} 
                               & $\beta_1$ &  8.8918    & -1986.5645    & 1986.5645     & 8.8918     &  2009.4069    \\ \cline{2-7} 
                               & $\beta_2$ &  30.0551    & -6835.3624    & 6835.3624     & 30.0551     & 7388.3095     \\ \cline{2-7} 
                               & $\tau$    &  4.9541    &  -28.5051   &  28.5051    &  -4.9541    & 30.4631     \\ \hline

            \multirow{4}{*}{3} & $\beta_0$ & 0.0139     & -12.8457    & 12.8457     &  -0.0139    & 14.9255     \\ \cline{2-7} 
                               & $\beta_1$ & 0.134     & -9.6703    &  12.1766    &  -0.1002    &  15.9566    \\ \cline{2-7} 
                               & $\beta_2$ &  0.3394    & -60.4861    & 60.4861     &  -0.3394    &  61.5831    \\ \cline{2-7} 
                               & $\tau$    &  0.3582    &  0.68   &  0.68    & 0.3582     &  0.7972    \\ \hline
            \end{tabular}
            \caption{VAR forecasting results for the structural NS factors.}
            \label{tab:structuralVAR}
        \end{table}