\documentclass{vegaarticle}

\def\fillandplacepagenumber{%
 \par\pagestyle{empty}%
 \vbox to 0pt{\vss}\vfill
 \vbox to 0pt{\baselineskip0pt
   \hbox to\linewidth{\hss}%
   \baselineskip\footskip
   \hbox to\linewidth{%
     \hfil\thepage\hfil}\vss}}

\usepackage{import}
\usepackage[capitalize]{cleveref}
\usepackage{pdflscape}
\usepackage{multirow}
\addbibresource{refs.bib}

\author{Vsevolod Zaostrovsky, Ivan Cherepakhin, Artemy Sazonov}
\title{Forecasting the Yield Curve: An Econometric Study}
\date{\today}
\nocite{*}
\graphicspath{{fig/}}
\jel{C53, G17, E47}

\begin{document}
    \maketitle

    \begin{abstract}{}
        This research study employs econometric analysis techniques to investigate the forecasting of the yield curve,
        analyze impulse response functions (IRFs), and detect structural breaks. Accurate forecasting of the yield curve
        is crucial for investors, policymakers, and risk managers in making informed decisions. The analysis of IRFs
        provides insights into the dynamic response of the yield curve to shocks in macroeconomic variables, allowing
        for a deeper understanding of the transmission mechanisms. Additionally, the study examines structural breaks in
        the yield curve associated with unpredictable events, providing valuable insights into shifts in market
        dynamics. By combining these three components, this research contributes to a broader understanding of the yield
        curve's behavior and its implications for financial markets and economic policies. The repository of the research 
        is available on \href{https://github.com/VsevolodZaostrovsky/FinancialEconometrics}{GitHub}.
    \end{abstract}

    \introduction
        The yield curve, depicting the relationship between time to maturity and yields on zero-coupon bonds, serves as
        a vital indicator of market expectations, economic conditions, and future monetary policy. Accurate forecasting
        of the yield curve has tremendous implications for investors, policymakers, and risk managers. Simultaneously,
        understanding the dynamic response of the yield curve to shocks and structural breaks enables a deeper
        comprehension of the underlying economic factors and their impact on financial markets.

        This paper aims to contribute to the existing literature by conducting a comprehensive econometric analysis that
        encompasses yield curve forecasting, impulse response analysis, and the investigation of structural breaks. By
        incorporating these three key components, this research seeks to shed light on the interplay between economic
        variables, forecast future yield curve movements, and detect shifts in the yield curve structure associated with
        unpredictable events.
        
        The first component of this study focuses on yield curve forecasting using econometric techniques, such as
        Vector Autoregression (VAR) models or Dynamic Nelson-Siegel (DNS) models. By leveraging historical data on
        zero-coupon yields and potential explanatory variables, the chosen model will provide forecasts for the yield
        curve over a specified time horizon. The accuracy of these forecasts will be rigorously evaluated using
        statistical measures such as mean absolute error (MAE) or root mean squared error (RMSE), allowing for an
        assessment of the model's predictive capabilities.
        
        In addition to forecasting, this research incorporates the analysis of impulse response functions (IRFs).
        Through estimating the dynamic response of the yield curve to shocks in relevant macroeconomic variables such as
        GDP growth, inflation, or monetary policy indicators, the IRFs provide insight into the transmission channels
        and the lagged effects of these shocks on the yield curve. This analysis will enhance our understanding of the
        interactions between the yield curve and important economic factors, contributing to the wider field of monetary
        policy and financial markets.
        
        Furthermore, we address the critical aspect of detecting and studying structural breaks in the yield curve.
        Unforeseen events, whether political, social, or economic in nature, can lead to significant shifts in the yield
        curve's structure. By employing robust econometric techniques, such as Chow tests, Bai-Perron tests, or
        Markov-switching models, this study will identify and examine these structural breaks. The timing, magnitude,
        and nature of the breaks will be analyzed, providing valuable insights into the factors driving the shifts and
        their implications for market dynamics.
        
        In summary, this research aims to offer a comprehensive analysis of the yield curve, incorporating yield curve
        forecasting, impulse response analysis, and the study of structural breaks. By combining these three aspects,
        we provide insights into the future movements of the yield curve, the dynamic response to macroeconomic shocks,
        and the detection of structural changes associated with unpredictable events. These findings hold significant
        implications for investors, policymakers, and market participants, ultimately contributing to a deeper
        understanding of macroeconomic dynamics and aiding in informed decision-making in financial markets.

    
    \import{./}{Data.tex}

    \import{./}{YC_Forecasting.tex}

    \import{./}{IRA.tex}

    \import{./}{SBA.tex}

    \import{./}{MCForecasting.tex}

    \section{Conclusion}
        During project 2 we researched the following:
        \begin{enumerate}
            \item Impulse response functions for the Nelson-Siegel factors;
            \item Structural breaks of the factors and their possible causes;
            \item MS-ARIMA forecasting of the factor time series.
        \end{enumerate} 
        We plan to improve our research in this are and contribute greatly to the field.


    \references
\end{document} 